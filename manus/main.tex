% Template for PLoS
% Version 3.4 January 2017
%
% % % %textcolor{red} % % % % % % % % % % % % % % % % % %
%
% -- IMPORTANT NOTE
%
% This template contains comments intended 
% to minimize problems and delays during our production 
% process. Please follow the template instructions
% whenever possible.
%
% % % % % % % % % % % % % % % % % % % % % % % 
%
% Once your paper is accepted for publication, 
% PLEASE REMOVE ALL TRACKED CHANGES in this file 
% and leave only the final text of your manuscript. 
% PLOS recommends the use of latexdiff to track changes during review, as this will help to maintain a clean tex file.
% Visit https://www.ctan.org/pkg/latexdiff?lang=en for info or contact us at latex@plos.org.
%
%
% There are no restrictions on package use within the LaTeX files except that 
% no packages listed in the template may be deleted.
%
% Please do not include colors or graphics in the text.
%
% The manuscript LaTeX source should be contained within a single file (do not use \input, \externaldocument, or similar commands).
%
% % % % % % % % % % % % % % % % % % % % % % %
%
% -- FIGURES AND TABLES
%
% Please include tables/figure captions directly after the paragraph where they are first cited in the text.
%
% DO NOT INCLUDE GRAPHICS IN YOUR MANUSCRIPT
% - Figures should be uploaded separately from your manuscript file. 
% - Figures generated using LaTeX should be extracted and removed from the PDF before submission. 
% - Figures containing multiple panels/subfigures must be combined into one image file before submission.
% For figure citations, please use "Fig" instead of "Figure".
% See http://journals.plos.org/plosone/s/figures for PLOS figure guidelines.
%
% Tables should be cell-based and may not contain:
% - spacing/line breaks within cells to alter layout or alignment
% - do not nest tabular environments (no tabular environments within tabular environments)
% - no graphics or colored text (cell background color/shading OK)
% See http://journals.plos.org/plosone/s/tables for table guidelines.
%
% For tables that exceed the width of the text column, use the adjustwidth environment as illustrated in the example table in text below.
%
% % % % % % % % % % % % % % % % % % % % % % % %
%
% -- EQUATIONS, MATH SYMBOLS, SUBSCRIPTS, AND SUPERSCRIPTS
%
% IMPORTANT
% Below are a few tips to help format your equations and other special characters according to our specifications. For more tips to help reduce the possibility of formatting errors during conversion, please see our LaTeX guidelines at http://journals.plos.org/plosone/s/latex
%
% For inline equations, please be sure to include all portions of an equation in the math environment.  For example, x$^2$ is incorrect; this should be formatted as $x^2$ (or $\mathrm{x}^2$ if the romanized font is desired).
%
% Do not include text that is not math in the math environment. For example, CO2 should be written as CO\textsubscript{2} instead of CO$_2$.
%
% Please add line breaks to long display equations when possible in order to fit size of the column. 
%
% For inline equations, please do not include punctuation (commas, etc) within the math environment unless this is part of the equation.
%
% When adding superscript or subscripts outside of brackets/braces, please group using {}.  For example, change "[U(D,E,\gamma)]^2" to "{[U(D,E,\gamma)]}^2". 
%
% Do not use \cal for caligraphic font.  Instead, use \mathcal{}
%
% % % % % % % % % % % % % % % % % % % % % % % % 
%
% Please contact latex@plos.org with any questions.
%
% % % % % % % % % % % % % % % % % % % % % % % %

\documentclass[10pt,letterpaper]{article}
\usepackage[top=0.85in,left=2.75in,footskip=0.75in]{geometry}

% amsmath and amssymb packages, useful for mathematical formulas and symbols
\usepackage{amsmath,amssymb}

% Use adjustwidth environment to exceed column width (see example table in text)
\usepackage{changepage}
\usepackage{xcolor}
% Use Unicode characters when possible
\usepackage[utf8x]{inputenc}

% textcomp package and marvosym package for additional characters
\usepackage{textcomp,marvosym}

% cite package, to clean up citations in the main text. Do not remove.
\usepackage{cite}

% Use nameref to cite supporting information files (see Supporting Information section for more info)
\usepackage{nameref,hyperref}

% line numbers
\usepackage[right]{lineno}

\usepackage{comment}

% toprule, midrule, bottomrule
\usepackage{booktabs}

\usepackage{floatrow}
% Table float box with bottom caption, box width adjusted to content
\newfloatcommand{capbtabbox}{table}[][\FBwidth]


% ligatures disabled
\usepackage{microtype}
\DisableLigatures[f]{encoding = *, family = * }

% color can be used to apply background shading to table cells only
%\usepackage[table]{xcolor}

% array package and thick rules for tables
\usepackage{array}
%Ad hoc for Table H
\usepackage{float}

\usepackage{xcolor}
\usepackage{subfigure}

% create "+" rule type for thick vertical lines
\newcolumntype{+}{!{\vrule width 2pt}}

% create \thickcline for thick horizontal lines of variable length
\newlength\savedwidth
\newcommand\thickcline[1]{%
  \noalign{\global\savedwidth\arrayrulewidth\global\arrayrulewidth 2pt}%
  \cline{#1}%
  \noalign{\vskip\arrayrulewidth}%
  \noalign{\global\arrayrulewidth\savedwidth}%
}

% \thickhline command for thick horizontal lines that span the table
\newcommand\thickhline{\noalign{\global\savedwidth\arrayrulewidth\global\arrayrulewidth 2pt}%
\hline
\noalign{\global\arrayrulewidth\savedwidth}}


% Remove comment for double spacing
%\usepackage{setspace} 
%\doublespacing

% Text layout
\raggedright
\setlength{\parindent}{0.5cm}
\textwidth 5.25in 
\textheight 8.75in

% Bold the 'Figure #' in the caption and separate it from the title/caption with a period
% Captions will be left justified
\usepackage[aboveskip=1pt,labelfont=bf,labelsep=period,justification=raggedright,singlelinecheck=off]{caption}
\renewcommand{\figurename}{Fig}

% Use the PLoS provided BiBTeX style
\bibliographystyle{plos2015}

% Remove brackets from numbering in List of References
\makeatletter
\renewcommand{\@biblabel}[1]{\quad#1.}
\makeatother

% Leave date blank
\date{}

% Header and Footer with logo
\usepackage{lastpage,fancyhdr,graphicx}
\usepackage{epstopdf}
\pagestyle{myheadings}
\pagestyle{fancy}
\fancyhf{}
\setlength{\headheight}{27.023pt}
\lhead{\includegraphics[width=2.0in]{PLOS-submission.eps}}
\rfoot{\thepage/\pageref{LastPage}}
\renewcommand{\footrule}{\hrule height 2pt \vspace{2mm}}
\fancyheadoffset[L]{2.25in}
\fancyfootoffset[L]{2.25in}
\lfoot{\sf PLOS}

%% Include all macros below

\newcommand{\lorem}{{\bf LOREM}}
\newcommand{\ipsum}{{\bf IPSUM}}


\usepackage{booktabs}
\usepackage{xcolor}
\usepackage[normalem]{ulem} % strikeout (\sout{text})

%% END MACROS SECTION


\begin{document}
\vspace*{0.2in}

% Title must be 250 characters or less.
\begin{flushleft}
{\Large
\textbf\newline{Lateral ventricle volume trajectories predict response inhibition in older age - a longitudinal brain imaging and machine learning approach} % Please use "sentence case" for title and headings (capitalize only the first word in a title (or heading), the first word in a subtitle (or subheading), and any proper nouns).
}
\newline
% Insert author names, affiliations and corresponding author email (do not include titles, positions, or degrees).

Astri J. Lundervold\textsuperscript{1},
Alexandra Vik\textsuperscript{1},
Arvid Lundervold\textsuperscript{2*}


\bigskip
\textbf{1} Department of Biological and Medical Psychology University of Bergen, 5009 Bergen, Norway
\\
\textbf{2} Mohn Medical Imaging and Visualization Centre, Department of Biomedicine, University of Bergen, Norway \\



% Insert additional author notes using the symbols described below. Insert symbol callouts after author names as necessary.
% 
% Remove or comment out the author notes below if they aren't used.
%
% Primary Equal Contribution Note
%\Yinyang These authors contributed equally to this work.

% Additional Equal Contribution Note
% Also use this double-dagger symbol for special authorship notes, such as senior authorship.
%\ddag These authors also contributed equally to this work.

% Current address notes
%\textcurrency Current Address: Dept/Program/Center, Institution Name, City, State, Country % change symbol to "\textcurrency a" if more than one current address note
% \textcurrency b Insert second current address 
% \textcurrency c Insert third current address

% Deceased author note
%\dag Deceased

% Group/Consortium Author Note
%\textpilcrow Membership list can be found in the Acknowledgments section.

% Use the asterisk to denote corresponding authorship and provide email address in note below.
* Corresponding author: arvid.lundervold@uib.no


\end{flushleft}

\section*{Abstract}

\vspace{-1mm}

{\bf Objective:} In a three-wave 6 yrs longitudinal study we investigated if the expansion of lateral ventricle (LV) volumes (regarded as a proxy for brain tissue loss) predicts third wave performance on a test of response inhibition (RI).
{\bf Participants and Methods:}  Trajectories
of left and right lateral ventricle volumes across the three waves were quantified using the longitudinal stream in Freesurfer. All participants ($N=74$;$48$ females;mean age $66.0$ yrs at the third wave) performed the Color-Word Interference Test (CWIT). Response time on the third condition of CWIT, divided into {\tt fast}, {\tt medium} and {\tt slow}, was used as outcome measure in a machine learning framework. Initially, we performed a linear mixed-effect (LME) analysis to describe subject-specific trajectories of the left and right LV volumes (LVV). These features were input to a multinomial logistic regression classification procedure, predicting individual belongings to one of the three RI classes. To obtain results that might generalize, we evaluated the significance of a $k$-fold cross-validated f1 score with a permutation test, providing a $p$-value that approximates the probability that the score would be obtained by chance. We also calculated a corresponding confusion matrix.
{\bf Results:} The LME-model showed an annual $\sim3.0$\% LVV increase. Evaluation of a cross-validated score using $500$ permutations gave an f1-score of $0.462$ that was above chance level ($p=0.014$). $56$\% of the fast performers were successfully classified. All these were females, and typically older than $65$ yrs at inclusion. For the true slow performers, those being correctly classified had higher LVVs than those being misclassified, and their ages at inclusion were also higher.
{\bf Conclusion:} Major contributions
were: (i) a longitudinal design, (ii) advanced brain imaging and segmentation procedures with longitudinal data analysis, and (iii) a data driven machine learning approach including cross-validation and permutation testing to predict behaviour, solely from the individual's brain ``signatures” (LVV trajectories).


\vspace{3mm}

[*]{Corresponding author: Arvid Lundervold, Department of Biomedicine, University of Bergen, Jonas Lies vei 91, 5009 Bergen, NORWAY,  E-mail: arvid.lundervold@uib.no, Phone: +47 55586353}

\emph{Short title}: Lateral ventricle volume trajectories and response inhibition \\


\newpage

\section{Introduction}
\linenumbers
Normal aging is associated with morphometric changes in several brain regions
and changes affecting cognitive function. The trajectories of age-related changes are, however, characterized by a large interindividual heterogeneity \cite{Nyberg2012}.  This is observed in studies of structural brain changes \cite{Fjell2013}, the rate and extent of cognitive changes \cite{Goh2012,Reuter-Lorenz2014} as well as brain-cognition relations in older age \cite{Nyberg2017,Vidal-Pineiro2018}, leaving some individuals with preserved cognitive function into old age, and others with a decline at a much younger age. In the severe end of the distribution, the most extensive tissue loss is associated with dementia, a syndrome defined by a severe decline in cognitive function \cite{Gale2018}. On the other end of the scale we find so-called ``superagers" \cite{Rogalski2013}. They show maintained cognitive function into old age \cite{Borelli2018}, with a corresponding preservation of brain structure over time \cite{Tampubolon2015, Nyberg2017}. This heterogeneity can be explained by several biological and genetic factors, as well as the many life-events and life-style factors that influence an individual through his or her life-time  \cite{Walhovd2014,Nyberg2019,Cabeza2018}. It has for example been shown that compensatory strategies developed through the life-time can slow down a cognitive decline in spite of a decline at a neuronal level \cite{Reuter-Lorenz2014}. This large number of unknown factors gives arguments for the relevance of a data driven approach when we investigate the relation between subject specific structural brain changes and cognitive function in the present study. 



Several previous studies have related changes in cognitive function to changes in specific regions and structures of the brain (e.g., \cite{McDonald2012,Aljondi2018,Gorbach2017,Pudas2018}). For example, prefrontal cortex has been linked to  global aspects of cognitive function like fluent intelligence \cite{Yuan2018} and to specific measures defined within the concept of executive function (e.g., \cite{Cardenas2011,Gunning-Dixon2003}). Executive function (EF) is of special interest in studies including older participants, as EF has been described as a hallmark of cognitive aging \cite{Buckner2004,Turner2012}. In the present study we have focused on response inhibition (RI), which is described as one of the core functional subcomponents of EF\cite{Friedman2017}, susceptible to impairment as part of normal cognitive aging \cite{Stuss2000,Adolfsdottir2017}. The close relation between RI and fluent intelligence \cite{Salthouse2003} and between fluid intelligence and various properties of brain structure \cite{Yuan2018} add to the interest of this EF subcomponent in relation to brain changes. 
The nature and empirical specificity of such relations between brain regions and RI is, however, still not clear.  Inconsistent results are reported and can at least partly be explained by individual differences in age-related volume changes across different brain regions \cite{Leong2017}, but also by what Salthouse et al. \cite{Salthouse2003} refer to as the "ability impurity" of EF tests. In fact, subfunctions of EF are most likely dependent on multiple, interconnected brain regions \cite{Friedman2017}. 
In the present study, we will therefore not use volume changes in specific brain tissue regions or structures as predictors of RI, but rather use trajectories of change in the lateral ventricle (LV) volumes as a proxy of age-related brain tissue loss. 
This because the lateral ventricular volumes (LVV) can be seen as a ``complement volume" of brain parenchyma since the intracranial volume (ICV)  
is regarded constant during adulthood and older age. \\


The choice of LV volumes (LVV) is further supported by studies describing the brain's fluid-filled ventricles as a biomarker of the aging brain \cite{Preul2006,Scahill2003}, and studies linking age-related ventricular expansion to changes in cognitive function at a subject-specific level \cite{Carmichael2007a, Carmichael2007b , Madsen2015}. 
A study by Todd et al. \cite{Todd2017} showed a strong linear relationship between LVV expansion and worsening of cognitive performance over a two-years period. The study assessed cognitive function by tests primarily designed to reveal symptoms of major neurocognitive disorders. Less is known about the longitudinal relationship between LVV expansion and more specified measures of cognitive functions that are prone to normal age-related changes. The eight year longitudinal study by Leong et al. \cite{Leong2017} is an exception. The study assessed the co-evolution of volumetric brain changes and cognitive function in a large group of healthy older adults (n=111, age range 56-83 yrs at baseline) including tests measures defined within specified cognitive domains. The results showed volumetric reduction of tissue across several brain regions, and that faster cerebral atrophy and ventricular expansion (at $3.56$\%/year) 
were associated with rapid decline in performance on tests of verbal memory and executive function. \\

The studies referred to above motivated us to further investigate the ability of predicting RI from LVV-derived biomarkers. 
Response inhibition is here defined from performance on the third condition of the Color-Word interference (CWIT) test, which is part of the Delis and Kaplan Executive Function Scale (D-KEFS) \cite{Delis2001}. 
Previous studies have controlled for the first two conditions of CWIT (color naming and word reading) in a linear regression model to obtain a more ``pure" measure of inhibition \cite{Adolfsdottir2014, Adolfsdottir2017}. 
In the present study we rather consider the complexity of the third condition as a strength, because it potentially gives a better match to the selected ``global" measure of tissue loss (i.e. LVV changes) and is also easier to interpret (RT in seconds).
Segmentation of the longitudinal 3D T1-weighted MRI recordings were used to measure the subject-specific trajectories of LVV change across the three study waves, and the RI performance at the third wave was included as an outcome variable, assuming that neuronal loss tends to precede cognitive decline in older age \cite{Reuter-Lorenz2014}. 

We see the application of (i) a longitudinal design, (ii) advanced brain imaging and segmentation procedures with longitudinal data analysis (LDA), and (iii) a data driven machine learning approach including cross-validation and permutation testing to predict behavior as the major contributions of the present study. Our aim was to predict RI performance (slow, medium, fast) solely from the individual's brain ``signatures” in terms of LVV trajectories, i.e. expressing and testing subject-specific brain-behavior relationships. 
By this, we wanted to contribute with methods and results that are likely more generalizable to unseen data than those obtained using ordinary linear regression or classification models applied to the full cohort without using hold-out or a train-test-split cross-validation procedure.
More specifically, after image segmentation we used a linear mixed-effect (LME) analysis similar to Leong et al. \cite{Leong2017} to describe and select characteristics of the subject-specific LVV trajectories of the left and right lateral ventricle.
From explorative data analysis, four features derived from the random effects component in the LME model were included in a multinomial logistic regression classification procedure, predicting individual belongings to one of three classes of performance level ({\tt slow}, {\tt medium} and {\tt fast}) on the RI test. A permutation test was used to evaluate the significance of a cross-validated F1-score to obtain results that may generalize to other samples (i.e. providing a $p$-value that approximates the probability that the score would be obtained by chance). From cross-validation, single subject predictions were obtained, enabling computation of a confusion matrix for better assessment and interpretation of our classifier performance.  

From this, we expected to confirm the volume expansion profiles of the lateral ventricles that Leong et al. \cite{Leong2017} reported from their statistical mixed effects model, as well as an association between LVV expansion and RI performance.


In the explorative data analysis we expected to reveal an age-related expansion of the lateral ventricle volumes \cite{Leong2017}, a slower age-related expansion of LVV in females than in males \cite{Chung2006,Hasan2014}, 
and that poor response inhibition performance is a more frequent in older age \cite{Stuss2000, Adolfsdottir2017}. By casting our brain and behavior measurements into a comprehensive classification framework, we hypothesized that model-based features characterizing the LVV  trajectories of an individual could act as predictors of \emph{his or her} RI performance. 
According to previous studies (see e.g. \cite{Nyberg2012}), the success-rate of this prediction was expected to scale with age, with better classification performance in the oldest segment of our cohort. \\ 




\section{Methods}

\subsection{Sample}
The study included a cohort of $74$ healthy middle-aged and older subjects ($48$ females and $26$ males). They were all part of a three-wave longitudinal study on cognitive aging, where subjects with a history of substance abuse, present neurological or psychiatric disorder, or other significant medical conditions were excluded from participation (see \cite{Espeseth2012, Lundervold2014} for more details). 
Their mean age was $59.9$ yrs (SD $7.3$), $63.3$ yrs ($7.2$) and $66.0$ yrs ($7.2$) for study wave $1$, $2$ and 
$3$, respectively, and their mean education was $13.94$ yrs ($2.9$). 
All the $74$ subjects provided MRI data across the three study-waves that could be successfully processed 
during cross-sectional Freesurfer segmentation without need of (subjective) manual editing, and were then run through the longitudinal stream of Freesurfer \cite{Dale1999} (details in Section 2.3). 
Results from the CWIT cognitive test of RI, administered as part of the third study-wave, were available for all the $74$ subjects. With an aim to investigate the opportunity and success of predicting  performance on a cognitive test from individual trajectories of volumetric brain measures, we decided to restrict the sample derived from our larger study of cognitive aging 
\cite{Espeseth2012, Lundervold2014} to those $74$ with a complete brain-cognition data set across the three waves. 


 An inspection of the neuropsychological test data from the three waves confirmed that none of the participants showed results indicating dementia. The test battery included two subtests from the Wechsler Abbreviated Scale of Intelligence (WASI, \cite{Wechsler1999}) administered in the first wave to estimate intellectual function, and the Mini Mental Staus Examination (MMSE, \cite{Folstein1975}) in waves $2$ and $3$.  All participants obtained a MMSE score  $\geq 25$, and their mean IQ score was $117.1$ ($10.2$). None of the participants reported or obtained a score on the second edition of the Beck Depression scale (BDI-II) \cite{Beck1987} that indicated depression. 

All participants signed an informed written consent form, and the study was approved by the Regional Committees for Medical and Health Research Ethics of Southern (study wave 1) and Western Norway (study wave 2 and 3).

 
\subsection{Response inhibition}

The total raw response-time (RT) score (in seconds) for correct responses on the third condition of the CWIT \cite{Delis2001}, performed as part of the third study wave, was included as the measure of RI. In this condition, subjects are requested to name the colors of color-words printed in incongruent colors (e.g., the the word ``red" printed in ``green") as fast and correct as possible. 
From this, it is assumed that the participant has to inhibit the more automatic response to read the word, commonly referred to as the Stroop effect. 
In the two preceding conditions of CWIT, the participants named a set of colours and read a set of color words. The third condition thus includes the effects of these two fundamental abilities \cite{Adolfsdottir2014}.   
Trained research assistants administrated the test in a quiet room designed for a neuropsychological examination.   


\subsection{MRI acquisition and brain segmentation}

Multi-modal MR imaging was performed on a 1.5 T GE Signa Echospeed scanner (MR laboratory, Haraldsplass Deaconess Hospital, Bergen) 
using a standard 8-channel head coil. 
Two consecutive T1-weighted 3D volumes were recorded from each subject (to improve SNR and brain segmentation) 
using a fast spoiled gradient echo (FSPGR) sequence 
(TE = $1.77$ ms; TR = $9.12$ ms; TI = $450$ ms; FA = $7^{\circ}$; FoV = $240 \times 240$ mm$^2$, image matrix = $256 \times 256 \times 124$;
voxel resolution = $0.94 \times 0.94 \times 1.40$ mm$^3$; 
TA = $6$:$38$ min). 
The same scanner (no upgrades) and T1-w 3D imaging protocol were used at each of the three study waves.\\

Brain segmentation and morphometric analysis across the three waves
was conducted using the {\tt Freesurfer} image analysis suite, version 5.3 (documented and freely available online from {\small \url{https://surfer.nmr.mgh.harvard.edu}}).
To extract reliable volume estimates and their trajectories (e.g. left and right lateral ventricles), the cross-sectionally processed images from the three study waves were subsequently run through the longitudinal stream \cite{Reuter2012} in Freesurfer.
Specifically, an unbiased within-subject template space and image is created using robust, inverse consistent registration \cite{Reuter2010}. 
Several processing steps, such as skull stripping, Talairach transforms, atlas registration as well as spherical surface maps and parcellations are then initialized with common information from the within-subject template, significantly increasing reliability and statistical power \cite{Reuter2012}. 
As a consequence of the longitudinal processing stream and within-subject registration, the estimated total intracranial volume (eTIV) for a given subject remains fixed across the three study waves. 
To illustrate data, processing stream, and results Fig. 1 depicts the longitudinal MRI original 
recordings ({\tt orig.mgz}) and the corresponding Freesurfer segmentations ({\tt aseg.mgz}) from one randomly selected participant at each of the three study waves. The age at the MRI examinations and corresponding left and right lateral ventricle volumes are shown along the time-line.

\vspace{3mm}


\begin{figure}[H]
\caption{
The longitudinal MRI recordings ({\tt orig.mgz}) and the corresponding Freesurfer segmentations ({\tt aseg.mgz}) from one of the participants at each of the three study waves. The age at the MRI examinations and corresponding left and right lateral ventricle volumes are given along the time-line.}
\label{fig 1}
\end{figure}


\vspace{3mm}


After running Freesurfer to its end on the collection of subjects, cross sectionally and followed by the longitudinal stream (several days on a standard Linux workstation), we obtained for each wave subject-specific Freesufer directories containing segmentation results (e.g. {\tt aseg.mgz} for inspection) and aggregated morphometric statistics (e.g. volume of left and right lateral ventricle and the intracranial volume, eTIV being constant for each subject, all in microliter). It was then easy to extract the volumetric data in tabular form for the whole cohort using a Python script. The subject's age at MRI examinations wave 1, 2 and 3 was derived from the 3D T1-w DICOM headers.
We further combined these variables with subject gender and RI reaction time at wave 3 to a single data frame, that also included the eTIV-normalized lateral ventricle volumes, $\frac{\mbox{LVV}}{\mbox{eTIV}}$. 
This {\tt Pandas} data frame was used in the following analyses.\\


\subsection{Statistical analyses}

\subsubsection{Identification of individual trajectories of LV volume changes} 

Mixed effects modelling was used to characterize individual trajectories of LVV change according to the following LME model equation: 

$$Vol_{ij}^H = \beta_0^H + \beta_1^H Age_{ij} + (b_{0i}^H + b_{1i}^H Age_{ij}) + \epsilon_{ij}^H,$$ 

where $H \in \{L, R\}$ denote hemisphere, $i$ is subject ($i = 1, \ldots,N = 74$)  and $j$ is wave ($j= 1,\ldots,n=3$). The response variable
$Vol_{ij}$ is volume of left (or right) lateral ventricle in subject $i$ at wave $j$, 
and $Age_{ij}$ (predictor) is age [years] of subject $i$ at wave $j$. 
The variables $\beta_0$ and $\beta_1$ are fixed effects model parameters, $b_{0i}$  and  $b_{1i}$ are random effects model parameters, and $\epsilon_{ij}$ is random residual errors,  with zero mean and constant variance $\delta = \epsilon 2$. 

\vspace{5mm}
Two features were derived from the LME model to characterize the individual LVV trajectories. 
The first (denoted {\tt b1i}) describes the steepness of individual volume trajectory, defined as the slope parameter in two-parameter family of random effects ($b_{0i}$, $b_{1i}$). 
The second feature (denoted {\tt Vdev}) describes an LVV deviation measure at baseline, and is defined as the difference at wave 1 between subject-specific LVV and the age-matched LVV expected from the cohort fixed effect regression line that is parameterized with ($\beta_0$, $\beta_1$).
For each of these features, one is selected from the right and one from the left hemisphere. 
This is motivated from expected mutual confirmation of LVV trajectories in the left and the right hemisphere, and also possible hemispheric differences as reported in previous studies (e.g. \cite{Trimarchi2013}). These four model-based features ({\tt b1iL}, {\tt b1iR}, {\tt VdevL}, {\tt VdevR}) were included as predictors in the further analyses (see Fig.2 for illustration). \\

\vspace{3mm}
\begin{figure}[H]
\caption{
Illustration (left hemisphere) of the subject-specific measures ({\tt b1iL}, {\tt b1iR}, {\tt VdevL}, {\tt VdevR}) of LVV trajectories obtained from the LME analysis. \\}
\label{fig 2}
\end{figure}
\vspace{3mm}


\subsubsection{Explorative data analysis} 

The distributions (i.e. kernel density estimation) and Pearson correlations between the six parameters: age at wave 3 ({\tt Age3}), the four LVV measures ({\tt b1iL}, {\tt b1iR}, {\tt VdevL}, {\tt VdevR}), and the reaction time from the RI measure in CWIT at wave 3 ({\tt RI3}) were calculated and presented separately for females and males as a comprehensive generalized pairs plot using the {\tt ggplot2} and {\tt GGally} packages in {\bf R} ver 3.5 (cf. Fig. 4). 


 \subsubsection{Prediction of response inhibition}
 
A classification approach with three categories of RI performance was used to investigate the predictive value of the four LVV measures. To generate such categories, the participants were divided into {\tt slow}, {\tt medium}, and {\tt fast} performers. First, a jittering procedure was used to eliminate RT ties, adding Gaussian $\mathcal{N}(\mu=0,\sigma^2)$ noise with $\sigma = 0.05$ to the integer valued reaction times, being in the range $[35, 102]$ (in seconds), such that each jittered RT was typically around $\pm 50$ ms from the measured one. 
A quantile-based discretization function was then used to compute four reaction time threshold values and corresponding reaction time intervals to obtain balanced classes, i.e. close to the same number of participants in each category (cf. Table 1).
\vspace{5mm}

 \begin{center}
\begin{tabular}{c|c|c|c|c}
{\bf RI label} & {\bf RT interval [sec]} & {\bf F} & {\bf M} & {\bf Total} \\
\hline
{\tt fast}   & $[ 34.9,\, 49.9 \rangle$ & 19 & 6 & 25 \\
{\tt medium} & $[ 49.9,\, 58.7 \rangle$ & 14 & 10 & 24 \\
{\tt slow}   & $[ 58.7, \, 102.1 \rangle$ & 15 & 10 & 25 \\
\hline
\end{tabular}
\end{center}

\noindent{Table 1: Definitions and characteristics of {\tt fast}, {\tt medium}, and {\tt slow} performers. RI = response inhibition; RT = reaction time; F = number of females; M = number of males.} 
\vspace{5mm}


For predicting category $y_i \in \{${\tt slow}, {\tt medium}, {\tt fast}$\}$  from explanatory variables
$X_i = $({\tt b1iL}$_i$, {\tt b1iR}$_i$, {\tt VdevL}$_i$, {\tt VdevR}$_i$) where $i \in \{1, \ldots, 74\}$
denote participant number $i$, we used a linear regularized logistic regression classifier as implemented in 
{\tt Logistic Regression} from the linear models in the {\tt scikit-learn} library for Python. 
Since we have a three-class problem, we used a multinomial version 
with the cross-entropy loss, a limited-memory Broyden–Fletcher–Goldfarb–Shanno (`lbfgs') solver,
L2 regularization with primal formulation, tolerance for stopping criteria $0.0001$, and let $500$ be the maximum number of iterations taken for the solver to converge. We fixed the value of parameter $C$ (the inverse of regularization strength in the algorithm) to be  $0.5$ 
in all our classification experiments without any hyperparameter tuning. \\
The best and most detailed description of the classifier being used is found 
in {\scriptsize  \url{ https://scikit-learn.org/stable/modules/generated/sklearn.linear_model.LogisticRegression.html}} and the references therein.


If a feature has a mean and variance that is orders of magnitude larger than others, it might dominate the objective function (cross-entropy loss) and make our classifier using L2 regularization unable to learn from other features correctly as expected. Such effect could be observed by assessing feature importance before and after preprocessing with a scaler (mean removal and variance scaling). To assure that
all features were centered around $0$ and have variance in the same order, the input data were preprocessed with {\tt scikit-learn}'s {\tt StandardScaler} obtaining zero mean and unit variance for each feature, also in every fold during cross validation (see below).


\subsubsection{Evaluation using $k$-fold cross validation with permutations} 

It is well known that learning the parameters of a prediction function and testing it on the same data is a methodological mistake. A sufficiently expressive model would just repeat the labels of the samples that it has just seen and could have a perfect score but fail to predict anything useful on yet unseen data, being a victim of {\it overfitting} and lack of {\it generalization abilities}.  
When performing our supervised machine learning experiments on labeled data (we let our complete dataset be denoted $(X,y)$ where $X$ is the 
$74 \times 4$ matrix of predictors and $y$ is the $74 \times 1$ vector of corresponding RI labels), a common practice is therefore to hold out part of the available data as a training set used for model estimation and the remaining samples as a test set for performance evaluation. 
However, by partitioning the available data into two sets (or, three when including a validation set for hyperparameter tuning), we drastically reduce the number of samples which can be used for learning the model, and the results can depend on a particular random choice for the pair of train and test datasets.
To ameliorate this problem, especially in small-sample size studies like ours, we used $k$-fold cross-validation (CV) to assess the prediction properties of our multinomial logistic regression classifier, such as performance scores (i.e. accuracy, precision, recall, and F1) and confusion matrices. 
In this procedure the dataset was split into $k$ smaller sets (stratified folds were made by preserving the percentage of samples for each class), and for each of the $k$ folds, a model was trained using $k-1$ of the folds as training data, and the estimated model was then applied on the remaining fold being used as a test dataset to compute performance scores.
The performance measure reported by $k$-fold cross-validation was the average of the values computed in the loop. In our analysis we report on average (`micro') F1 score, calculated globally by counting the total true positives ($TP$), false negatives ($FN$) and false positives ($FP$),  
and interpreted as a weighted average of the precision $= TP/(TP+FP)$ and the recall $= TP/(TP+FN)$, 
i.e. $F1 = 2$ (precision $\times$ recall) / (precision + recall).

In order to test if a classification score was significant, a technique of repeating several times the 
$k$-fold CV classification procedure after randomizing the labels was used, 
i.e. evaluating the significance of a cross-validated score with permutation testing.
By this means, a $p$-value approximates the probability that the score would be obtained by chance is given by the percentage of runs for which the score obtained was greater than the classification score obtained in the first place (cf. Fig. 4). In our experiments we used the 
{\tt permutation\_test\_score} function in {\tt scikit-learn} with $k=5$, and $500$ permutations, yielding a  $p$-value $= (C + 1) / (500 + 1)$, where $C$ is the number of permutations whose score $\geq$ the true score.
The minimum $p$-value is $1/(500+1) \approx 0.002$ corresponding to the case where the classifier is so good that none of the classifiers with shuffled labels has a better score, and the worst value is $1.0$.
The {\tt permutation\_test\_score} computations returned the true score without permuting labels, an array of scores for each permutation, and 
the $p$-value described above. These are reported in the Results section (cf. Fig. 4).
To further assess our model, we generated cross-validated estimates for each of the $74$ data points in $X$ (with corresponding RI label $y$) using the same $k$-fold cross-validation and standard scaling as described above. Mapping each data point in the input to the
prediction that was obtained for that element when it was in the test set, was done for diagnostic purposes - illustrating typical confusion matrices and scores obtained from the model - not for measuring generalization error as was previously done in the permutation testing. Finally, we computed the $3 \times 3$ {\it confusion matrix} using the true labels (Observed RI in Fig. 7) versus the classification labels returned from the cross validation prediction (Predicted RI in Fig. 7).

\vspace{3mm}
 
\noindent All analyses were implemented as {\tt Jupyter notebooks} 
using {\tt Python} (3.6), {\tt Numpy} (1.14), {\tt Pandas} (0.23), {\tt Matplotlib} (3.0), 
{\tt Statsmodels} (0.9), {\tt Scikit-learn} (0.20), and {\tt rpy2} (2.9) with {\tt R} (3.5) and packages {\tt lme4}, {\tt ggplot2} and {\tt GGally} for producing Figures 3, 4, and 6. These notebooks with corresponding datasets as {\tt .csv} files were tested to run under {\tt Anaconda} on both MacOS 10.14, Windows 10, and Ubuntu 18.04 platforms and will be available on GitHub [{\small \url{https://github.com/arvidl/lvv-ri}}].


\section{Results}

\subsection {Three wave changes in lateral ventricular volumes}

From the linear mixed-effect (LME) model used to investigate the age-related evolution of the ventricular volumes Figures 2a and 2b show the fixed (fat unbroken line) and random effects (thin line segments) calculated from the LVV modeling for the left and right hemispheres, respectively.
Figures 3a and 3b show the corresponding data and LME-model fitted to the eTIV-normalized LVV values.
The fixed effects regression line shows expansion of LV volumes (or, eTIV-normalized LVVs) with increasing age. From the fixed effect model we found an overall cohort volume increase of $429$~$\mu$L/year 
for the left side LVV, and $426$~$\mu$L/year 
for the right side. With a mean LVV in left hemisphere of $14994$ [$\mu$L] at inclusion, this represents
an annual $\approx2.9$ \% increase in left side LVV, and with a mean LVV in right hemisphere of $13777$ [$\mu$L], this represents an annual $\approx3.1$ \% right side LVV increase. 
Visual inspection reveals a trend towards a steeper slope for the older participants in the cohort. Furthermore, the fixed effects regression line was less steep than ordinary linear least squares regression line (fat broken line), demonstrating the effect of the LDA approach that takes into account the {\it dependencies} between the subject-specific measures across the three study waves. 


\vspace{3mm}
\begin{figure}[H]
\caption{
 Subject-specific longitudinal lateral ventricle volumes versus age in left ({\bf a}) and right ({\bf b}) hemisphere shown as color-coded spaghetti plots across the three study waves. For left and right hemisphere the random effects, estimated from the linear mixed-effect model $Vol_{ij} = \beta_0 + \beta_1 \mbox{Age}_{ij} + (b_{0i} + b_{1i} \mbox{Age}_{ij}) + \epsilon_{ij}$, are depicted as thin line segments in black superimposed on the color-coded line plots. The thick regression line in black represents the estimated fixed effect, and the broken line represents ordinary linear least squares regression (OLS) line.
Subject-specific longitudinal {\it eTIV-normalized} lateral ventricle volumes versus age in left ({\bf c}) and right ({\bf d}) hemisphere, respectively, are shown as color-coded spaghetti plots across the three study waves. Here, a linear mixed-effect model was applied and fitted to the eTIV-normalized data.\\}
\label{fig 3}
\end{figure}
\vspace{3mm}


\subsection {Explorative data analysis}

Figure 4 shows the kernel density estimated distributions of age, the four volume measures, and the response inhibition performance ({\tt RI}), and their pair-wise Pearson correlations, with separate panels for the use of non-normalized LVVs with respect to the subject's ICV (a), and the eTIV-normalized LVVs (b). \\

The gender effects are shown by presenting the results separately for females ($n$=$48$) and males ($n$=$26$). The LVV-derived measures for females were shifted towards the lower end of the distribution compared to males, while the gender-specific distributions were less different for age and RI in both using native LVVs and eTIV-normalized LVVs.  The Pearson correlations were strong between the left ({\tt b1iL}) and right ({\tt b1iR}) slope measures in (a) $r$=$0.94$ (and also for the eTIV-normalized LVVs $r=0.93$ in (b)), and between the two deviance measures {\tt VdevL} and {\tt VdevR}: $r=0.89$ in (a), $r=0.87$ in (b). Statistically significant correlation was found, for females only, between {\tt RI3} and {\tt b1iL} ($r = 0.48$) and between {\tt RI3} and {\tt b1iR} ($r = 0.53$). For the eTIV-normalized LVVs similar correlations were found (in females only). Age at wave 3 was moderately correlated with the four lateral ventricular features in females. In males these correlations were generally lower and non-significant. This was the case for both native LVVs and for eTIV-normalized LVVs. Due to the small qualitative difference between the use of native LV volumes and eTIV-normalized volumes observed in the exploratory data analyses (cf. Figure 4, and Figure 5), we performed our machine learning classification experiments using features derived from the native LV volumes, only. \\
 
\vspace{3mm}
\begin{figure}[H]
\caption{
Generalized pairs plot depicting the kernel density estimated empirical distributions of each of the six variables and Pearson correlations between age, the four LVV trajectory measures and response inhibition. (a)~Non-normalized LVVs. (b)~eTIV-normalized LVVs. The graphs and correlations are given separately for females (in red) and males (in green). {\tt Age3} = age of participant at study wave 3; ${\tt b1iL}$ = LVV steepness measure, left hemisphere; ${\tt b1iR}$ = LVV steepness measure, right hemisphere; ${\tt VdevL}$ = LVV deviance measure, left hemisphere; ${\tt VdevR}$ = LVV deviance measure, right hemisphere; {\tt RI3} = response inhibition reaction time at study wave 3. \\}
\label{fig 4}
\end{figure}
\vspace{3mm}




\subsection{Predicting response-inhibition from LVV trajectories}

The four LME-based features selected to characterize the non-normalized LVV trajectories, i.e. slope of LVV change ({\tt b1i}) and the LVV deviation at the time of inclusion ({\tt Vdev}), from both the right and from the left hemispheres, were used to compute our cross-validated score to predict level of RI. Figure 5 shows the results from our simulation experiments using iteratively fitted multinominal logistics regression models ($n=500$ permutations) to assess the significance of the {\tt f1}-score. The vertical green dotted line represent our cross-validation classification score of $0.462$ and shows that the score is significantly better ($p = 0.014$) than the $0.333$ chance level (black dotted line).

\vspace{3mm}
\begin{figure}[H]
\caption{
Result from the simulation experiments assessing the significance of a $5$-fold cross-validated score (f1) with $500$ permutations using multinomial logistic regression. The predictors
are $X = \{${\tt b1iL}, {\tt b1iR}, {\tt VdevL}, {\tt VdevR}$\}$ and the classes are the three levels of RI reaction times, 
$y = \{${\tt slow}, {\tt medium}, {\tt fast}$\}$. \\}
\label{fig 5}
\end{figure}
\vspace{3mm}


The results from the $k$-fold cross-validation procedure is presented in Table 2. The precision (positive predictive value) is higher than the recall (sensitivity) for the {\tt slow} and {\tt medium} RI classes, but lower than the recall score for {\tt fast} performers. The overall slightly best 
{\tt f1}-score was obtained for the {\tt fast} performers. The {\tt fast} performers also had a recall score that was higher than any other score metric, regardless level of performance.


\begin{table}[H]
\begin{center}
\begin{tabular}{lrrrc}
\toprule
{} &  {\bf f1-score} &  {\bf precision} &  {\bf recall} &  {\bf support} \\
\midrule

{\tt slow}        &    0.4255 &     0.4545 &  0.4000 &     25 \\
{\tt medium}       &    0.4545 &     0.5000 &  0.4167 &     24 \\
{\tt fast}         &    0.4912 &     0.4375 &  0.5600 &     25 \\
{\it micro avg}    &    0.4595 &     0.4595 &  0.4595 &     74 \\
{\it macro avg}    &    0.4571 &     0.4640 &  0.4589 &     74 \\
{\it weighted avg} &    0.4571 &     0.4635 &  0.4595 &     74 \\
\bottomrule
\end{tabular}
\end{center}
\end{table}
\vspace{-5mm}


\noindent {\bf Table 2}: Predictions from each split of cross-validation, generating cross-validated estimates for each input data point using multinomial logistic regression. F1-measure, precision, recall, and support for each class are computed. The predictors
are $X = \{${\tt b1iL}, {\tt b1iR}, {\tt VdevL}, {\tt VdevR}$\}$ and the classes are the three levels of RI reaction times, 
$y = \{${\tt slow}, {\tt medium}, {\tt fast}$\}$. The reported averages include {\it micro average} (averaging the total true positives, false negatives and false positives), {\it macro average} (averaging the unweighted mean per label), and {\it weighted average} (averaging the support-weighted mean per label). The support is the number of occurrences of each class in $y_{\mbox{\scriptsize true}}$.\\

\vspace{5mm}
Figure 6 illustrates the $74$ subject-specific trajectories color-coded with the {\bf observed (true) RI label} for left hemisphere (a) and the right hemisphere (c). The same  $74$ subject-specific trajectories are then color-coded with the {\bf predicted RI label} for left hemisphere (b) and the right hemisphere (d). The most successful classification, in both hemispheres, is for the {\tt fast} performers as illustrated by the red line-segments. The {\tt slow} performers, shown by the blue line-segments, seem to be most successfully classified if their age was above $65$ years at inclusion. \\

\vspace{3mm}
\begin{figure}[H]
\caption{
Plots showing the {\bf observed} RI labels (leftmost two panels, for left (a) and right hemisphere (c), respectively) and the {\bf predicted} RI labels (rightmost two panels, for left (b) 
and right hemisphere (d), respectively) for each of the $74$ subjects in the cohort. When a given trajectory in (a) or (c) changes its color as it occur in (b) or (d), that subject is misclassified; otherwise he or she is correctly classified with respect to RI performance. \\}
\label{fig 6}
\end{figure}
\vspace{3mm}



The $3 \times 3$ {\it confusion matrix} (CM) compares the true labels (Observed RI in Fig. 7) versus the classification labels returned from the cross validation prediction (Predicted RI in Fig. 7). We have also computed CM cell-specific information about gender ratio ({\tt F/M}), number of participants older than 65 years at baseline ({\tt Age1} $> 65$), and volume means in microliters of left and right lateral ventricle ({\tt Vol1L} and {\tt Vol1R}), respectively, at baseline. The confusion matrix in Fig. 7 shows that $\frac{14}{25}$ = $56$\% of the fast performers were correctly classified, all were females, and five of these were older than $65$ years at inclusion. Only one participant older than $65$ at inclusion who was a {\tt fast} performer was misclassified. We also found that the correctly classified {\tt fast} performers were among those who had the smallest LVVs at baseline. The {\tt fast} performers that were misclassified also had larger LVVs at inclusion. For the true {\tt slow} performers, those being correctly classified had higher LVVs than those being misclassified, and their age at inclusion were also higher. A relatively high proportion ($40$ \%) of the {\tt slow} performers were misclassified as {\tt fast}. In this group there were more females than males, few were older than $65$ years at inclusion, and their LVVs were substantially lower ($<50$ \%) than those being correctly classified.


\vspace{3mm}
\begin{figure}[H]
\caption{
The $3 \times 3$ {\it confusion matrix} 
computed for the {\tt slow}, {\tt medium} and {\tt fast} RI labels returned from the cross validation prediction with our multinomial logistic regression model compared with the co-occurrences of the true (observed) RI labels. The diagonal cells are those representing correctly classified subjects (number of occurrences in each cells are given as $N$), and these cells are shaded in blue. Off-diagonal cells represents various events of misclassification.  Observed/predicted co-occurrences are also accompanied, for each cell, with corresponding information about gender ratio ({\tt F/M}), confirmed age at inclusion larger than 65 years ({\tt Age1} $> 65$), and volume means in microliters of left and right lateral ventricle ({\tt Vol1L} and {\tt Vol1R}), respectively, at time of subject inclusion in the study.}  
\label{fig 7}
\end{figure}
\vspace{3mm}

\section{Discussion}
The present study used an LME model to describe, visualize, and design four features characterizing subject-specific LVV trajectories: slope of his or her volume change across the three study waves and a measure of age-related deviance between cohort LVV and subject LVV at inclusion in the study. These LME-based features where then input as predictors of level of RI performance using a linear regularized multinomial logistic regression classifier within a machine learning framework incorporating $k$-fold cross validation and permutation testing.
Visual inspection of the LME results revealed an approximately linear age-related expansion of the lateral ventricle volumes over the six years period of observations. The exploratory data analysis showed that distributions of all four LVV features were characterized by gender differences, and that significant correlations between response inhibition performance, age and the LVV slope measure were mostly restricted to the female part of the sample.
A cross-validated score predicted performance defined within the three RI classes with a mean classification {\tt f1}-score that was only moderately good ($0.462$), but clearly better than chance level ($p<0.02$). A confusion matrix revealed that {\tt fast} performers were most successfully predicted. Furthermore, the group of successfully classified {\tt fast} performers included only females, participants with the smallest LVVs at baseline, and all but one {\tt fast} performer older than $65$ at inclusion. For those being successfully classified as {\tt slow} performers, 67\% were older than $65$ years at inclusion and their LVVs were higher that those being misclassified within this {\tt slow} RI class.

The results confirmed that healthy aging is associated with a slight expansion of the lateral ventricular system. This finding further supports arguments for using information about volume of brain's fluid-filled ventricles as an imaging-derived biomarker in studies of the aging brain and cognition \cite{Leong2017,Preul2006,Scahill2003,Carmichael2007a, Carmichael2007b,Madsen2015,Todd2017}.
Interestingly, the present study estimated an annual fixed-effect increase in LVVs at $\approx3$\% in the cohort (slightly larger in right hemisphere compared to the left), close to the ventricular expansion ($3.56$\%/year) reported in the study by Leong et al.  \cite{Leong2017}, a study that inspired the present work.
In addition to the LME-modelling approach, our contribution relates to data from {\it three} study waves being analyzed (Leong et al. reported results from $111$ subjects in a two-waves-study), and that we were taking the analysis one step further, bringing the data into a predictive machine learning (ML) framework. 
By this, we obtained results that could be applied at a single case level, being obtained with a method ($k$-fold cross validation) that are aimed to have generalization abilities and applicable to yet-unseen data. In this ML context  we could show that different classes of RI performance ({\tt slow}, {\tt medium} and {\tt fast}) could be predicted from the LVV trajectories with an accuracy and f1-score that was moderate but clearly above chance level, and further emphasize the importance of gender and age illustrated by the explorative data analysis and the extended confusion matrix. 


The confusion matrix showed that all {\tt fast} performers who were correctly classified were females, and that the overall percentage of correctly classified females ($54$\%) was higher than for males ($31$\%). These results demonstrate the importance of gender, also shown in the explorative data analysis. Here, the Pearson correlations between level of RI performance and the two LVV slope measures were much stronger in the female part of the sample. By this, our results were similar to the results reported by Aljondi et al.~\cite{Aljondi2018} in a female-only sample. 
In \cite{Aljondi2018} they used the same Freeurfer longitudinal stream image analysis to obtain atrophy estimates, and a similar linear regression analysis to model brain-cognition changes as in our study. Gender differences in rate of LVV expansion have been reported in previous studies, indicating a slower expansion in females than males \cite{Chung2006, Hasan2014}. Results from our explorative data analysis suggested that the rate of expansion to be age-related. The slope of the LVV trajectories were lower in females than males in the younger age groups but shifted to a higher value in females in the oldest age groups. The lack of consistent results across studies may thus be related to age differences in the samples. For example, the slower progression of volume expansion in females than males reported 
by Chung et al.~\cite{Chung2006} and 
Hasan and collaborators~\cite{Hasan2014} were based on data from a younger sample than in the present study (i.e. subjects in their $40$s and between $18$ and $59$ years, respectively). 

The importance of including participants with age $>65$ years was illustrated by the extended confusion matrix being computed in our study. This matrix showed that all but one {\tt fast} performer who were $>65$ at baseline were correctly classified. We may speculate if these {\tt fast} performers of age $>65$, and with relatively small LVVs (about $1/3$ of LVV for those with {\tt slow} RI performance) represent what Rogalski and collaborators referred to as ``superagers”~\cite{Rogalski2013}, and that their LVV trajectories can serve (or contribute) as predictors of preserved brain function into old age - at least in females. Future and more extensive longitudinal studies, regarding sample size, duration and age span, are therefore indeed warranted.  

We will also underscore the results obtained from the {\tt slow} RI performers. Our measures of LVV trajectories correctly classified eight of twelve ($67$\%) slow performers aged $>65$ years.  
If we assume that their slow performance on the RI test reflects a preclinical sign of a Mild Cognitive Impairment (MCI), the results would have important clinical implications. Previous studies have shown that more than $50$\% of MCI patients are expected to progress and convert to dementia within five years~(e.g. \cite{Gauthier2006}). Although speculative, our methods and results may be relevant to efforts in obtaining better and more accurate diagnostic and monitoring tools for brain health in older adults. Individual change in the rate of ventricular expansion such as LVVs, using measures from linear or non-linear mixed effect models estimated in large cohorts, could act as a sensitive measure of an early stage of a neurodegenerative disease~\cite{Jack2004}.

The somewhat low number of participants ($N=74$) make us unable to state firm and general conclusions. A larger sample 
could improve our classifications scheme, incorporating the LME-model estimation within the cross-validation loop, reducing the amount of ``data leakage” (i.e. the training set and the test set sharing information) in the present study.
Furthermore, the value of including of a larger number and diverse set of predictors have been well demonstrated in studies based on theoretical models considering brain maintenance and cognitive reserves (e.g.,\cite{Nyberg2017, Reuter-Lorenz2014}. These models emphasize the importance of life-events \cite{Reuter-Lorenz2014, Stern2018}, a richer set of imaging information using multimodal MRI~\cite{Arbabshirani2017, Nyberg2019} and 
PET~\cite{Nyberg2016,Nevalainen2015}. Together, this provides strong arguments for sharing data (and code) across research groups \cite{Calhoun2016} and use of predictive models and methods within modern machine learning frameworks~\cite{Arbabshirani2017}, in order to support open science and reproducible research.\\

\subsection{Conclusion}We showed that a set of four LME-derived measures of LVV trajectories across three study waves gave a fairly good prediction of RI performance, confirming the role of lateral ventricle volumes as an imaging-based biomarker of cognitive function in older adults. 
Our major contributions are the application of (i) a three wave longitudinal design, (ii) advanced brain imaging and segmentation procedures with longitudinal data analysis, and (iii) a data driven machine learning approach including cross-validation and permutation testing to predict RI performance solely from the individual's brain ``signatures” (LVV trajectories). Future studies should further investigate this avenue
regarding brain-behavior relationships in older age, using larger
sample sizes, gender balance and age spans, than were available in the present study. \\


\section*{Acknowledgments}
The first study-wave was financially supported by the Research Council of Norway (grant 154313/V50), the second and third study-wave from the Western Norway Regional Health Authority (grants \#911397 and \#911687). We also acknowledge the Western Norway Regional Health Authority for funding the Freesurfer Longitudinal Stream (grant \#911995) and the Bergen Research Foundation for support through the project  "Computational medical imaging and machine learning - methods, infrastructure and applications". Finally, we thank colleagues and participants in the longitudinal study on cognitive aging.
\nolinenumbers


\begin{thebibliography}{10}

\bibitem{Nyberg2012}
Nyberg L, L{\"o}vd{\'e}n M, Riklund K, Lindenberger U, B{\"a}ckman L.
\newblock Memory aging and brain maintenance.
\newblock Trends in Cognitive Science. 2012 May;16(5):292-305. doi:10.1016/j.tics.2012.04.005. 

\bibitem{Fjell2013}
Fjell AM, Westlye LT, Grydeland H, Amlien I, Espeseth T, Reinvang I, et~al.
\newblock Critical ages in the life course of the adult brain: nonlinear
  subcortical aging.
\newblock Neurobiology of Aging. 2013 Oct;34(10):2239-47. doi:10.1016/j.neurobiolaging.2013.04.006.

\bibitem{Goh2012}
Goh JO, An Y, Resnick SM.
\newblock Differential trajectories of age-related changes in components of
  executive and memory processes.
\newblock Psychology and Aging. 2012 Sep;27(3):707-19. doi:10.1037/a0026715. 

\bibitem{Reuter-Lorenz2014}
Reuter-Lorenz PA, Park, DC. 
\newblock How does it STAC up? Revisiting the scaffolding theory of aging and cognition
\newblock Neuropsychology Review. 2014 Sep;24(3):355-70. 
doi:10.1007/s11065-014-9270-9.



\bibitem{Vidal-Pineiro2018}
Vidal-Pi{\~{n}}eiro D, Sneve MH, Nyberg LH, Mowinckel AM, Sederevicius D, Walhovd KB, et~al.
\newblock Maintained Frontal Activity Underlies High Memory Function Over 8
  Years in Aging.
\newblock Cerebral Cortex. 2018 Aug 23. doi:10.1093/cercor/bhy177.

\bibitem{Nyberg2017}
Nyberg L.
\newblock Neuroimaging in aging: brain maintenance.
\newblock F1000Res. 2017 Jul 25;6:1215. doi:10.12688/f1000research.11419.1.
  
\bibitem{Gale2018}
Gale SA, Acar D, Daffner KR.
\newblock Dementia.
\newblock The American Journal of Medicine. 2018 Oct;131(10):1161-1169. doi:10.1016/j.amjmed.2018.01.022. 

\bibitem{Rogalski2013}
Rogalski EJ, Gefen T, Shi J, Samimi M, Bigio E, Weintraub S, et~al.
\newblock Youthful memory capacity in old brains: anatomic and genetic clues
  from the Northwestern SuperAging Project.
\newblock Journal of Cognitive Neuroscience. 2013 Jan;25(1):29-36. doi:0.1162/jocn\_a\_00300.

\bibitem{Borelli2018}
Borelli WV, Schilling LP, Radaelli G, Ferreira LB, Pisani L, Portuguez MW, et~al.
\newblock Neurobiological findings associated with high cognitive performance
  in older adults: a systematic review.
\newblock International Psychogeriatrics. 2018 Apr 18:1-13. doi:10.1017/S1041610218000431.

\bibitem{Tampubolon2015}
Tampubolon G.
\newblock Cognitive Ageing in Great Britain in the New Century: Cohort
  Differences in Episodic Memory.
\newblock PloSOne. 2015 Dec 29;10(12):e0144907. doi:10.1371/journal.pone.0144907.

 
\bibitem{Cabeza2018}
Cabeza R, Albert M, Belleville S, Craik FIM, Duarte A, Grady CL, Lindenberger U, Nyberg L, Park DC, Reuter-Lorenz PA, Rugg MD, Steffener J, Rajah MN.
\newblock Maintenance, reserve and compensation: the cognitive neuroscience of healthy ageing.
\newblock Nature reviews. Neuroscience. 2018 Nov;19(11):701-710. doi:10.1038/s41583-018-0068-2.

\bibitem{Walhovd2014}
Walhovd KB, Fjell AM, Espeseth T.
\newblock Cognitive decline and brain pathology in aging--need for a dimensional, lifespan and systems vulnerability view.
\newblock Scandinavian Journal of Psychology. 2014 Jun;55(3):244-54. doi:10.1111/sjop.12120.

\bibitem{Nyberg2019}
Nyberg L, Pudas S.
\newblock Successful Memory Aging.
\newblock Annual Review of Psychology. 2019 Jun;70:219–43. doi:10.1146/annurev-psych-010418-103052.

\bibitem{McDonald2012}
McDonald CR, Gharapetian L, McEvoy LK, Fennema-Notestine C, Hagler DJ, Holland
  D, et~al.
\newblock Relationship between regional atrophy rates and cognitive decline in
  mild cognitive impairment.
\newblock Neurobiology of Aging. 2012 Feb;33(2):242-253. doi:10.1016/j.neurobiolaging.2010.03.015

\bibitem{Aljondi2018}
Aljondi R, Szoeke C, Steward C, Yates P, Desmond P.
\newblock A decade of changes in brain volume and cognition.
\newblock Brain Imaging and Behavior. 2018 May;9. doi:10.1007/s11682-018-9887-z

\bibitem{Gorbach2017}
Gorbach T, Pudas S, Lundquist A, OrÀdd G, Josefsson M, Salami A, et~al.
\newblock Longitudinal association between hippocampus atrophy and
  episodic-memory decline.
\newblock Neurobiology of Aging. 2017 Mar;51:167-176. doi:10.1016/j.neurobiolaging.2016.12.002.

\bibitem{Pudas2018}
Pudas S, Josefsson M, Rieckmann A, Nyberg L.
\newblock Longitudinal Evidence for Increased Functional Response in Frontal
  Cortex for Older Adults with Hippocampal Atrophy and Memory Decline.
\newblock Cerebral Cortex. 2018 Mar 1;28(3):936-948. doi:10.1093/cercor/bhw418.

\bibitem{Yuan2018}
Yuan P, Voelkle MC, Raz N.
\newblock Fluid intelligence and gross structural properties of the cerebral cortex in middle-aged and older adults: A multi-occasion longitudinal study.
\newblock NeuroImage. 2018 May 15;172:21-30. doi:10.1016/j.neuroimage.2018.01.032.

\bibitem{Cardenas2011}
Cardenas VA, Chao LL, Studholme C, Yaffe K, Miller BL, Madison C, et~al.
\newblock Brain atrophy associated with baseline and longitudinal measures of cognition.
\newblock Neurobiology of Aging. 2011 Apr;32(4):572-80. doi:10.1016/j.neurobiolaging.2009.04.011.

\bibitem{Gunning-Dixon2003}
Gunning-Dixon FM, Raz N.
\newblock Neuroanatomical correlates of selected executive functions in middle-aged and older adults: a prospective {MRI} study.
\newblock Neuropsychologia. 2003 Jan;41(14):1929-1941. doi:10.1016/S0028-3932(03)00129-5

\bibitem{Buckner2004}
Buckner RL.
\newblock Memory and Executive Function in Aging and {AD}.
\newblock Neuron. 2004 Sep;44(1):195-208. doi:10.1016/j.neuron.2004.09.006

\bibitem{Turner2012}
Turner GR, Spreng RN.
\newblock Executive functions and neurocognitive aging: dissociable patterns of brain activity.
\newblock Neurobiology of Aging. 2012 Apr;33(4):826.e1-13. doi:10.1016/j.neurobiolaging.2011.06.005.

\bibitem{Friedman2017}
Friedman NP, Miyake A.
\newblock Unity and diversity of executive functions: Individual differences as a window on cognitive structure.
\newblock Cortex. 2017 Jan;86:186-204. doi:10.1016/j.cortex.2016.04.023.

\bibitem{Stuss2000}
Stuss DT, Alexander MP.
\newblock Executive functions and the frontal lobes: a conceptual view.
\newblock Psychological research. 2000 63:289-298. doi:10.1007/s004269900007.

\bibitem{Adolfsdottir2017}
Ad{\'o}lfsd{\'o}ttir S, Wollschlaeger D, Wehling E, Lundervold AJ.
\newblock Inhibition and Switching in Healthy Aging: A Longitudinal Study.
\newblock Journal of International Neuropsychological Society. 2017 Jan;23(1):90-97. doi:10.1017/S1355617716000898.

\bibitem{Salthouse2003}
Salthouse TA, Atkinson TM, Berish DE.
\newblock Executive Functioning as a Potential Mediator of Age-Related Cognitive Decline in Normal Adults.
\newblock Journal of Experimental Psychology: General. 2003;132(4):566-594. doi:10.1037/0096-3445.132.4.566

\bibitem{Leong2017}
Leong RLF, Lo JC, Sim SKY, Zheng H, Tandi J, Zhou J, et~al.
\newblock Longitudinal brain structure and cognitive changes over 8 years in an East Asian cohort.
\newblock NeuroImage. 2017 Feb 15;147:852-860. doi:10.1016/j.neuroimage.2016.10.016.

Shook BA, Lennington JB, Acabchuk RL, Halling M, Sun Y, Peters J, et~al.
\newblock Ventriculomegaly associated with ependymal gliosis and declines in barrier integrity in the aging human and mouse brain.
\newblock Aging cell. 2014 Apr;13(2):340-50. doi:10.1111/acel.12184.

\bibitem{Preul2006}
Preul C, Hund-Georgiadis M, Forstmann BU, Lohmann G.
\newblock Characterization of cortical thickness and ventricular width in normal aging: a morphometric study at 3 Tesla.
\newblock Journal of Magnetic Resonance Imaging : JMRI. 2006 Sep;24:513--519. doi:10.1002/jmri.20665.

\bibitem{Scahill2003}
Scahill RI, Frost C, Jenkins R, Whitwell JL, Rossor MN, Fox NC.
\newblock A longitudinal study of brain volume changes in normal aging using serial registered magnetic resonance imaging.
\newblock Archives of Neurology. 2003 Jul;60(7):989--94. doi:10.1001/archneur.60.7.989


\bibitem{Carmichael2007b}
Carmichael OT, Kuller LH, Lopez OL, Thompson PM, Dutton RA, Lu A, et~al.
\newblock Cerebral ventricular changes associated with transitions between normal cognitive function, mild cognitive impairment, and dementia.
\newblock Alzheimer Disease and Associated Disorders. 2007 21:14-24. doi:10.1097/WAD.0b013e318032d2b1

\bibitem{Carmichael2007a}
Carmichael OT, Kuller LH, Lopez OL, Thompson PM, Dutton RA, Lu A, et~al.
\newblock Ventricular volume and dementia progression in the Cardiovascular Health Study.
\newblock Neurobiology of Aging. 2007 Mar;28:389-397. doi:10.1016/j.neurobiolaging.2006.01.006

\bibitem{Madsen2015}
Madsen SK, Gutman BA, Joshi SH, Toga AW, Jack CR Jr, Weiner MW, et~al.
\newblock Mapping ventricular expansion onto cortical gray matter in older adults.
\newblock Neurobiology of Aging. 2015 Jan;36 Suppl 1:S32-41. doi:10.1016/j.neurobiolaging.2014.03.044.

\bibitem{Todd2017}
Todd KL, Brighton T, Norton ES, Schick S, Elkins W, Pletnikova O, et~al.
\newblock Ventricular and Periventricular Anomalies in the Aging and Cognitively Impaired Brain.
\newblock Frontiers in Aging Neuroscience. 2018 Jan 12;9:445. doi:10.3389/fnagi.2017.00445.

\bibitem{Delis2001}
Delis DC, Kaplan E, Kramer JH.
\newblock Delis-Kaplan Executive Function System.
\newblock San Antonio, TX: The Psychological Corporation.; 2001.

\bibitem{Adolfsdottir2014}
Ad{\'o}lfsd{\'o}ttir S, Ha{\'a}sz J, Wehling E, Ystad M, Lundervold A, Lundervold AJ.
\newblock Salient measures of inhibition and switching are associated with frontal lobe gray matter volume in healthy middle-aged and older adults.
\newblock Neuropsychology. 2014 Nov;28(6):859-69. doi:10.1037/neu0000082. 


\bibitem{Chung2006}
Chung SC, Tack GR, Yi JH, Lee B, Choi MH, Lee BY, et~al.
\newblock Effects of gender, age, and body parameters on the ventricular volume of Korean people.
\newblock Neuroscience letters. 2006 Mar;395:155-158. doi:10.1016/j.neulet.2005.10.066

\bibitem{Hasan2014}
Hasan KM, Moeller FG, Narayana PA.
\newblock DTI-based segmentation and quantification of human brain lateral ventricular CSF volumetry and mean diffusivity: validation, age, gender effects and biophysical implications.
\newblock Magnetic Resonance Imaging. 2014 Jun;32(5):405-12.
doi:10.1016/j.mri.2014.01.014.

\bibitem{Espeseth2012}
Espeseth T, Christoforou A, Lundervold AJ, Steen VM, Le Hellard S, Reinvang I.
\newblock Imaging and cognitive genetics: the Norwegian Cognitive NeuroGenetics sample.
\newblock Twin Research and Human Genetics. 2012 Jun;15(3):442-52. doi:10.1017/thg.2012.8.

\bibitem{Lundervold2014}
Lundervold  AJ, Wollschläger D, Wehling E.
\newblock Age- and sex-related changes in episodic memory function in middle-aged and older individuals. \newblock Scandinavian Journal of Psychology. 2014 Jun; 55, 225–232. doi: 10.1111/sjop.12114.

\bibitem{Dale1999}
Dale AM, Fischl B, Sereno MI.
\newblock Cortical Surface-Based Analysis I: Segmentation and Surface Reconstruction.  
\newblock NeuroImage. 1999 9(2):179-194. doi:10.1006/nimg.1998.0395

\bibitem{Wechsler1999}
Wechsler D.
\newblock Wechsler Abbreviated Scale of intelligence. WASI. Manual.
\newblock The Psychological Corporation; 1999.

\bibitem{Folstein1975}
Folstein MF, Folstein SE, McHugh PR.
\newblock "Mini-mental state". A practical method for grading the cognitive state of patients for the clinician.
\newblock Journal of Psychiatric Research. 1975 Nov;12(3):189-198.

\bibitem{Beck1987}
Beck, A. T., Steer, R. A., Brown, G. K. 
\newblock. Beck Depression Inventory (2nd ed.). 
\newblock San Antonio, TX: Psychological Corporation; 1987.

\bibitem{Reuter2012}
Reuter, M., Schmansky, N.J., Rosas, H.D., Fischl, B. 
\newblock Within-Subject Template Estimation for Unbiased Longitudinal Image Analysis. 
\newblock Neuroimage 2012 Jul 16;61(4):1402-18. doi:10.1016/j.neuroimage.2012.02.084. 

\bibitem{Reuter2010}
Reuter, M., Rosas, H.D., Fischl, B.
\newblock Highly Accurate Inverse Consistent Registration: A Robust Approach. 
\newblock Neuroimage.  2010 Dec;53(4):1181-96. doi:10.1016/j.neuroimage.2010.07.020.


\bibitem{Trimarchi2013}
Trimarchi F, Bramanti P, Marino S, Milardi D, Di Mauro D, Ielitro G, Valenti B, Vaccarino G, Milazzo C, Cutroneo G. 
\newblock MRI 3D lateral cerebral ventricles in living humans: morphological and morphometrical age-, gender-related preliminary study.
\newblock Anatomical Science International. 2013 88:61–69. 
doi 10.1007/s12565-012-0162-x

\bibitem{Salthouse2012a}
Salthouse TA.
\newblock Does the direction and magnitude of cognitive change depend on initial level of ability?
\newblock Intelligence. 2012 jul;40(4):352--361. doi:10.1016/j.intell.2012.02.006


\bibitem{Gauthier2006}
Gauthier S, Reisberg B, Zaudig M, Petersen RC, Ritchie K, Broich K, et~al.
\newblock Mild cognitive impairment.
\newblock Lancet (London, England). 2006 Apr;367:1262-1270. doi:10.1016/S0140-6736(06)68542-5.

\bibitem{Jack2004}
Jack CR, Shiung MM, Gunter JL, O’Brien PC, Weigand SD, Knopman DS, et al. 
\newblock Comparison of different MRI brain atrophy rate measures with clinical disease progression in AD. \newblock Neurology. 2004 Feb;62:591–600. doi.org/10.1212/01.WNL.0000110315.26026.EF.


\bibitem{Habeck2016}
Habeck C, Razlighi Q, Gazes Y, Barulli D, Steffener J, Stern Y.
\newblock Cognitive Reserve and Brain Maintenance: Orthogonal Concepts in Theory and Practice.
\newblock Cerebral Cortex. 2016; 1;27(8):3962-3969. doi:10.1093/cercor/bhw208


\bibitem{Stern2018}
Stern Y, Gazes Y, Razlighi Q, Steffener J, Habeck C.
\newblock A task-invariant cognitive reserve network.
\newblock {NeuroImage}. 2018 Sep;178:36-45. doi:10.1016/j.neuroimage.2018.05.033.

\bibitem{Arbabshirani2017}
Arbabshirani MR, Plis S, Sui J, Calhoun VD.
\newblock Single subject prediction of brain disorders in neuroimaging: Promises and pitfalls.
\newblock NeuroImage.  2017 Jan 15;145(Pt B):137-165. doi:10.1016/j.neuroimage.2016.02.079.



\bibitem{Nyberg2016}
Nyberg L, Karalija N, Salami A, et al.
\newblock Dopamine D2 receptor availability is linked to hippocampal-caudate functional connectivity and episodic memory. 
\newblock Proceedings of the National Academy of Sciences of the United States of America. 2016;113(28):7918–23. doi:10.1073/pnas.1606309113.

 
  
\bibitem{Nevalainen2015}
\newblock Nevalainen N, Riklund K, Andersson M, et al.
\newblock COBRA: A prospective multimodal imaging study of dopamine, brain structure and function, and cognition. 
\newblock Brain Research. 2015;1612:83–103. doi:10.1016/j.brainres.2014.09.010.


\bibitem{Calhoun2016}
Calhoun VD, Sui J.
\newblock Multimodal Fusion of Brain Imaging Data: A Key to Finding the Missing
  Link(s) in Complex Mental Illness.
\newblock Biological Psychiatry: Cognitive Neuroscience and Neuroimaging. 2016
  May;1(3):230--244. doi:10.1016/j.bpsc.2015.12.005


  
%\bibitem{Lundervold2019}




\end{thebibliography}


\end{document}



